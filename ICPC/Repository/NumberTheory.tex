\nsection{Number Theory}

\subsection{Goldbach conjecture}
\vspace{-5pt}
\begin{itemize}[noitemsep]
  \item All number $\geq$ 6 can be written as sum of 3 $primes$
  \item All even number > 2 can be written as sum of 2 $primes$
\end{itemize}
\vspace{-15pt}

\subsection{Prime numbers distribution}
Amount of primes approximately $\frac{n}{\ln(n)}$

\subsection{Sieve of Eratosthenes}
Numbers up to $2e8$ \\
\addfile{../Codes/NumberTheory/BitSieve.cpp} 
To factorize divide $x$ by $factor[x]$ until is equal to $1$ \\
\addfile{../Codes/NumberTheory/FactorizeSieve.cpp} 
Use it if you need a huge amount of $phi[x]$ up to some N \\
\addfile{../Codes/NumberTheory/PhiSieve.cpp} 
\vspace{-15pt}

\subsection{Phi of euler}
\addfile{../Codes/NumberTheory/Phi.cpp}

\subsection{Miller-Rabin}
\addfile{../Codes/NumberTheory/MillerRabin.cpp}

\subsection{Pollard-Rho}
\addfile{../Codes/NumberTheory/PollardRho.cpp}

\subsection{Amount of divisors}
\addfile{../Codes/NumberTheory/AmountOfDivisors.cpp}

% 2 3 5 7 11 13 17 19 23 29
% 31 37 41 43 47 53 59 61 67 71
% 73 79 83 89 97 101 103 107 109 113
% 127 131 137 139 149 151 157 163 167 173
% 179 181 191 193 197 199 211 223 227 229
% 233 239 241 251 257 263 269 271 277 281
% 283 293 307 311 313 317 331 337 347 349
% 353 359 367 373 379 383 389 397 401 409
% 419 421 431 433 439 443 449 457 461 463
% 467 479 487 491 499 503 509 521 523 541
% 547 557 563 569 571 577 587 593 599 601
% 607 613 617 619 631 641 643 647 653 659
% 661 673 677 683 691 701 709 719 727 733
% 739 743 751 757 761 769 773 787 797 809
% 811 821 823 827 829 839 853 857 859 863
% 877 881 883 887 907 911 919 929 937 941
% 947 953 967 971 977 983 991 997 1009 1013
% 1019 1021 1031 1033 1039 1049 1051 1061 1063 1069
% 1087 1091 1093 1097 1103 1109 1117 1123 1129 1151
% 1153 1163 1171 1181 1187 1193 1201 1213 1217 1223
% 1229 1231 1237 1249 1259 1277 1279 1283 1289 1291
% 1297 1301 1303 1307 1319 1321 1327 1361 1367 1373
% 1381 1399 1409 1423 1427 1429 1433 1439 1447 1451
% 1453 1459 1471 1481 1483 1487 1489 1493 1499 1511
% 1523 1531 1543 1549 1553 1559 1567 1571 1579 1583
% 1597 1601 1607 1609 1613 1619 1621 1627 1637 1657
% 1663 1667 1669 1693 1697 1699 1709 1721 1723 1733
% 1741 1747 1753 1759 1777 1783 1787 1789 1801 1811
% 1823 1831 1847 1861 1867 1871 1873 1877 1879 1889
% 1901 1907 1913 1931 1933 1949 1951 1973 1979 1987
% 1993 1997 1999 2003 2011 2017 2027 2029 2039 2053
% 2063 2069 2081 2083 2087 2089 2099 2111 2113 2129
% 2131 2137 2141 2143 2153 2161 2179 2203 2207 2213
% 2221 2237 2239 2243 2251 2267 2269 2273 2281 2287
% 2293 2297 2309 2311 2333 2339 2341 2347 2351 2357
% 2371 2377 2381 2383 2389 2393 2399 2411 2417 2423
% 2437 2441 2447 2459 2467 2473 2477 2503 2521 2531
% 2539 2543 2549 2551 2557 2579 2591 2593 2609 2617
% 2621 2633 2647 2657 2659 2663 2671 2677 2683 2687
% 2689 2693 2699 2707 2711 2713 2719 2729 2731 2741
% 2749 2753 2767 2777 2789 2791 2797 2801 2803 2819
% 2833 2837 2843 2851 2857 2861 2879 2887 2897 2903
% 2909 2917 2927 2939 2953 2957 2963 2969 2971 2999
% 3001 3011 3019 3023 3037 3041 3049 3061 3067 3079
% 3083 3089 3109 3119 3121 3137 3163 3167 3169 3181
% 3187 3191 3203 3209 3217 3221 3229 3251 3253 3257
% 3259 3271 3299 3301 3307 3313 3319 3323 3329 3331
% 3343 3347 3359 3361 3371 3373 3389 3391 3407 3413
% 3433 3449 3457 3461 3463 3467 3469 3491 3499 3511
% 3517 3527 3529 3533 3539 3541 3547 3557 3559 3571 

% Big primes near to $10^n$ \\
% 9941 9949 9967 9973 10007 10009 10037 10039 10061 10067 10069 10079
% 99961 99971 99989 99991 100003 100019 100043 100049 100057 100069
% 999959 999961 999979 999983 1000003 1000033 1000037 1000039
% 9999943 9999971 9999973 9999991 10000019 10000079 10000103 10000121
% 99999941 99999959 99999971 99999989 100000007 100000037 100000039 100000049
% 999727999 999997489 999994007 999988243 999992153 999999893 999999929 999999937 1000000007 1000000009 1000000021 1000000033 1070777777
% 1000002193 1000008223 1000009999 1000027163

\subsection{Bézout's identity} 
\ncomment{$a_1 * x_1 + a_2 * x_2 + ... + a_n * x_n = g $}
\ncomment{$g = \gcd(a_1, a_2, ..., a_n)$ }

\subsection{GCD}
\ncomment{$a \leq b; \gcd(a + k, b + k) = \gcd(b - a, a + k)$}
% \addfile{../Codes/NumberTheory/GCD.cpp}
\vspace{-15pt}

\subsection{LCM}
\ncomment{$x = p * lcm(a_1, a_2, ..., a_k) + q, 0 \leq q \le lcm(a_1, a_2, ..., a_k)$}
\ncomment{$x \pmod{a_i} \equiv q \pmod{a_i}$ as $a_i \mid lcm(a_1, a_2, ..., a_k)$ \\}
% \addfile{../Codes/NumberTheory/LCM.cpp}
\vspace{-15pt}

\subsection{Euclid}
\addfile{../Codes/NumberTheory/Euclid.cpp}

\subsection{Chinese remainder theorem}
\addfile{../Codes/NumberTheory/ChineseRemainderTheorem.cpp}